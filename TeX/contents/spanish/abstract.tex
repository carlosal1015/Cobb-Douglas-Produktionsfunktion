\selectlanguage{spanish}

\begin{abstract}
La función de producción Cobb-Douglas es un enfoque neoclásico para estimar la función de producción de un país y proyectar de esta manera su crecimiento económico esperado. Para representar las relaciones entre la producción obtenida se utiliza las variaciones de los insumos como el capital ($K$) y el trabajo ($L$), a los que más tarde se añadió la tecnología, llamada también productividad total de los factores ($PTF$). Es una función de producción frecuentemente utilizada en Economía.
%https://assets.aeaweb.org/asset-server/files/9434.pdf

El origen de la función Cobb-Douglas se encuentra en la observación empírica de la distribución de la renta nacional total de Estados Unidos entre el capital y el trabajo. De acuerdo a lo que mostraban los datos, la distribución se mantenía relativamente constante a lo largo del tiempo. Concretamente el trabajo se llevaba un 70\% y el capital un 30\%. De esta forma, la función Cobb-Douglas representa una relación en donde las proporciones de trabajo y capital con respecto al producto total son constantes.%\pygment{python}{module}
\end{abstract}