\section{El modelo de Solow sobre el crecimiento}

\begin{itemize}
	\item Desarrollar un simple marco para los causes próximos y la mecánica del crecimiento económico y los países cruzados %TODO: income
	diferencias.
	\item El modelo de Solow-Swan llamado después de Robert Solow y Trevor Swan.
	\item Antes del modelo de Solow, la aproximación más común al crecimiento económico se construyó sobre el modelo de Harrod-Domar.
	\item El modelo de Harrod-Domar enfatizó los potenciales aspectos disfuncionales del crecimiento: por ejemplo, cómo el crecimiento podría ir mano a mano con el incremento del desempleo.
	\item El modelo de Solow demostró por qué el modelo de Harrod--Domar no fue un lugar atractivo para iniciar.
	\item En el centro del modelo de crecrimiento de Solow está la función de producción agregada neoclásica.
\end{itemize}

\begin{itemize}
	\item La economía cerrada, con un único final bueno.
	\item El tiempo discreto corriendo en un horizonte infinito, el tiempo es indexado por $t=0,1,2,\ldots$.
	\item La economía es inhabitada por un largo número de %TODO: households
	, y por ahora los %TODO: households
	no serán optimizados.
	\item Para fijar ideas, asuma que los %TODO: households
	son idénticos, así la economía admite un representativo %TODO: households
	\item Asuma que los hogares guardan una fracción exógena constante $s$ de tu ingreso disponible.
	\item Asuma que todas las formas tienen acceso a la misma función de producción: la economía admite una \textbf{firma representativa}, con un representante (o agregado) de la función de producción.
	\item La función de producción agregada para el único final bueno es
	\begin{equation}\label{eq:aggregate}
	Y\left(t\right)=F\left[K\left(t\right),L\left(t\right),A\left(t\right)\right].
	\end{equation}
	\item Asuma que el capital es el mismo como el producto final de la economía, pero usado en el proceso de producción de más productos.
	\item $A\left(t\right)$ es un cambio de la función de producción~\eqref{eq:aggregate}. Noción amplia de la tecnología.
	\item Suposición principal: la tecnología es \textbf{gratuita}, este está disponible como no excluible, no tiene un producto rival.
\end{itemize}

\begin{suppose}[Continuidad, diferenciabilidad, positiva y productos marginales decrecientes, y retornos de escala constante]
La función de producción $F\colon\mathds{R}^{3}_{+}\rightarrow\mathds{R}$ es dos veces diferenciable en $K$ y $L$ y satisface
\begin{align*}
F_{K}\left(K,L,A\right)\equiv\frac{\partial F}{\partial K}>0,\quad F_{L}\left(K,L,A\right)\equiv\frac{\partial F\left(\cdot\right)}{\partial L}>0,\\
F_{KK}\left(K,L,A\right)\equiv\frac{\partial^{2}F\left(\cdot\right)}{\partial K^{2}}<0,\quad F_{LL}\left(K,L,A\right)\equiv\frac{\partial^{2}F\left(\cdot\right)}{\partial L^{2}}<0.
\end{align*}
Es más, $F$ exhibe un retorno de escala constante en $K$ y $L$.
\end{suppose}
Asuma que $F$ exhibe un retorno a ecala constante en $K$ y $L$, es decir, es homogénea linealmente (homogénea de grado $1$) en estos dos variables.
\begin{definition}
Sea $K$ un entero. La función $g\colon\mathds{R}^{K+2}\rightarrow\mathds{R}$ es homogénea de grado $m$ en $x\in\mathds{R}$ e $y\in\mathds{R}$ si y solo si \[ g\left(\lambda x,\lambda y, z\right)=\lambda^{m}g\left(x,y,z\right)\text{ para todo }\lambda\in\mathds{R}_{+}\text{ y }z\in\mathds{R}^{K}. \]
\end{definition}
\begin{theorem}[Euler]
Suponga que $g\colon\mathds{R}^{K+2}\rightarrow\mathds{R}$ es continuamente diferenciable en $x\in\mathds{R}$ e $y\in\mathds{R}$, con derivadas parciales denotada por $g_{x}$ y $g_{y}$ y es homogénea de grado $m$ en $x$ e $y$. Entonces \[ mg\left(x,y,z\right)=g_{x}\left(x,y,z\right)x+g_{y}\left(x,y,z\right)y\quad\forall x\in\mathds{R},y\in\mathds{R}\text{ y }z\in\mathds{R}^{K}. \] Más aún, $g_{x}\left(x,y,z\right)$ y $g_{y}\left(x,y,z\right)$ son por sí mismo homogéneas de grado $m-1$ en $x$ e $y$.
\end{theorem}

\subsection{Estructura del mercado, dotaciones y limpieza del mercado}
\begin{itemize}
	\item Asumiremos que los mercados son competitivos, así que nuestro será un prototipo del equilibrio de un modelo general competitivo.
	\item Los hogares poseen todo el trabajo, que suministran inelásticamente.
\end{itemize}