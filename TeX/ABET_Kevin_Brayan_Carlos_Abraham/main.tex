% arara: pdflatex
\documentclass[
    a4paper,% Tamaño 14
    11pt, % 11 puntos de letra
    fleqn % alineación de la ecuación la izquierda
    %twocolumn% A dos columnas
]{article}
\usepackage[T1]{fontenc}% Permite copiar el texto del pdf.
\usepackage[utf8]{inputenc}% Admite las tildes en compiladores de LaTeX antiguos, Hoy ya es obsoleto, pero emplear por compatibliidad.
\usepackage[spanish]{babel} % Idioma español
\spanishdatedel
\usepackage{mathtools,amssymb,dsfont} % Matemática
\usepackage{lipsum}
\title{
    La función de Cobb-Douglas
    \thanks{Clifford}
}

\author{
    Kevin\and
    Brayan\and
    Carlos
}

\begin{document}

\maketitle

\section{Introducción}
\lipsum[1]

\section{Algunas menciones}

\subsection{Cobb-Douglas y la función de producción ACMS}

Partiendo de la función Cobb-Douglas
\begin{equation}%\label{eq:1}
Y = bL^{k}C^{1-k},
\end{equation}
donde:
\begin{itemize}
	\item $b$ representa el factor total de productividad,
	\item $Y$ la producción total,
	\item $L$ el trabajo,
	\item $C$ el capital.
\end{itemize}
Esta función fue generalizada siendo expresada de la manera siguiente:
\begin{equation}%\label{eq:2}
f = \gamma{x}_{1}^{\alpha_{1}}\cdots x_{n}^{\alpha_{n}},
\end{equation}
donde $\gamma$ es una constante positiva y $\alpha_{1},\ldots,\alpha_{n}$ son constantes no cero.

Se dice que una función de producción $f$ es $d$--homogénea o homogénea de degradación $d$, si:
\begin{equation}%\label{eq:3}
f\left(tx_{1},\ldots,tx_{n}\right) = t^{d}f\left(x_{1},\ldots,x_{n}\right),
\end{equation}
Se mantiene para cada $t\in\mathbb{R}$ en la función previamente definida.

Una función \emph{homogénea de degradación uno} es llamado como \emph{linealmente homogéneo}.

Si $d>1$, la función homogénea mostrará un crecimiento a escala, caso contrario cuando $d<1$ mostrará un decrecimiento a escala.

Arrow, Chenery, Minhas y Solow(ACMS) introdujeron una función de producción de dos factores:
\begin{equation}%\label{eq:4}
Q = F\cdot(aK^r + (1-a)L^r)^{1/r},
\end{equation}
donde:
\begin{itemize}
	\item $Q$ representa el resultado,
	\item $F$ el factor de producción,
	\item $a$ el parámetro compartido,
	\item $k,L$ los factores de producción primario,
	\item $r=(s- 1)/s$ , $s=1/(1-r)$ como la substitución de elasticidad.
\end{itemize}

La función de producción generalizada de ACMS se define:
\begin{equation}\label{eq:5}
f\left(x\right) = \gamma\sum_{i=1}^{n} ({{a}_{i}^{p}x_{i}^{p}})^{d/p},x=\left(x_{1},\ldots,x_{n}\right)\in D\subset\mathbb{R}_{+}^{n},
\end{equation}
con $a_{1},\ldots,a_{n},\gamma,p\neq0$, donde $d$ es la degradación de homogeneidad.

Cabe resaltar que la \emph{función de producción homotética} tiene la siguiente expresión como una función de producción:
\begin{equation}\label{eq:6}
f(x) = F\left(h(x_1,\ldots,x_n\right),
\end{equation}
donde F es una función estrictamente creciente y $h\left(x_1,\ldots,x_n\right)$ es una función homogénea de cualquier degradación d. La \emph{función de producción homotética} de la forma:
\begin{equation}\label{eq:7}
f\left(x\right) = \gamma\sum_{i=1}^{n} ({{a}_{i}^{p}x_{i}^{p}})^{d/p},\quad(\text{resp.},\quad f(x)=F(x_{1}^{\alpha_1}\cdots x_{n}^{\alpha_n}),
\end{equation}
es llamada \emph{función de producción ACMS generalizada homotética}.
\end{document}