\documentclass[11pt]{article}
\usepackage{amsmath, amssymb, amscd, amsthm, amsfonts}
\usepackage{graphicx}
\usepackage{hyperref}

\oddsidemargin 0pt
\evensidemargin 0pt
\marginparwidth 40pt
\marginparsep 10pt
\topmargin -20pt
\headsep 10pt
\textheight 8.7in
\textwidth 6.65in
\linespread{1.2}

\title{Función de producción de Cobb Douglas \footnote{Profesor reponsable, Clifford Torres, docente de la especialidad de Matemática, ctorresp@uni.edu.pe}}
\author{Kevin Fernandez\footnote{Código 20182228G, kfernandezh@uni.pe.}\and Brayan Torres\footnote{Código 20171339G, btorresa@uni.pe} \and Carlos Aznarán\footnote{Código 20162720C, caznaranl@uni.pe}
}
\date{}

\newtheorem{theorem}{Theorem}
\newtheorem{lemma}[theorem]{Lemma}
\newtheorem{conjecture}[theorem]{Conjecture}
\newtheorem{obs}{Observación}
\newtheorem{ejem}{Ejemplo}

\newcommand{\rr}{\mathbb{R}}

\newcommand{\al}{\alpha}
\DeclareMathOperator{\conv}{conv}
\DeclareMathOperator{\aff}{aff}
\renewcommand{\figurename}{Figura}


\begin{document}

\maketitle


\section{Resumen} % The start of a new section

La función de producción Cobb Douglas es un enfoque neo clásico para estimar la función de producción de un país y proyectar de esta manera su crecimiento económico esperado. Para representar las relaciones entre la producción obtenida se utiliza las variaciones de los insumos capital ($K$) y trabajo ($L$), a los que más tarde se añadió la tecnología, llamada también productividad total de los factores ($PTF$). Es una función de producción frecuentemente utilizada en Economía.

El origen de la función Cobb Douglas se encuentra en la observación empírica de la distribución de la renta nacional total de Estados Unidos entre el capital y el trabajo. De acuerdo a lo que mostraban los datos, la distribución se mantenía relativamente constante a lo largo del tiempo. Concretamente el trabajo se llevaba un 70\% y el capital un 30 \%. De esta forma, la función Cobb Douglas representa una relación en donde las proporciones de trabajo y capital con respecto al producto total son constantes.

\section{Abstract}
The function of production Cobb Douglas has a neoclassic focus to estimate the function of production of a country and to proyect the expected economyc growth.\\To represent the relationships between the production obtained, it uses the variations of capital (K) and labor (L) inputs, to which technology was later added, also called total factor productivity (PTF). It is a production function frequently used in Economics.\\
The origin of the Cobb Douglas function is found in the empirical observation of the distribution of the total national income of the United States between capital and labor. According to what the data showed, the distribution remained relatively constant over time. Specifically, work took 70\%\ and capital 30\%\ . In this way, the Cobb Douglas function represents a relationship in which the proportions of labor and capital with respect to the total product are constant.

\section{Introducción}
Este proyecto hace referencia a la función de producción de Cobb Douglas, siendo este publicado en 1928 por Charles Cobb y Paul Douglas, quienes realizaron un estudio en el que se modeló el crecimiento de la economía estadounidense.\\
Para este proyecto e desarollara una aplicación con una base de datos como un caso particular; pero previo a su aplciación, se realizara una posible forma de cómo Charles Cobb y Paul Douglas llegaron a la formulación algebacica de la función. Al finalizar, se comparará la solución exacta de la ecuación con la obtenida por el método de mínimos cuadrados.

\section{Función de producción de Cobb-Douglas}
Para el análisis matematico de la función , es necesario describir los factores involucrados en el modelo.

\subsection{Función de producción}
Es la relación entre las cantidades máximas de productos que una empresa puede fabricar mediante el uso de diversas cantidades de insumos. Las múltiples funciones de producción están representadas por la combinación de factores de insumo (tecnología, capital, trabajo entre otros)
Una función de produccion se expresa como la Ecuación(1) siguiente:\

\begin{equation}\label{ec1}
    P=f(K,L,I) 
\end{equation}
Donde:\\
$P$ es la cantidad de producción\\
$K,L,I$ son los insumos


%\bibliographystyle{alpha}
%\bibliography{references} % see references.bib for bibliography management

\end{document}
