\selectlanguage{english}

\maketitle

\begin{abstract}
The function of production Cobb-Douglas has a neoclassic focus to estimate the function of production of a country and to proyect the expected economyc growth.\\To represent the relationships between the production obtained, it uses the variations of capital ($K$) and labor ($L$) inputs, to which technology was later added, also called total factor productivity ($PTF$). It is a production function frequently used in Economics.

The origin of the Cobb Douglas function is found in the empirical observation of the distribution of the total national income of the United States between capital and labor. According to what the data showed, the distribution remained relatively constant over time. Specifically, work took 70\% and capital 30\%. In this way, the Cobb Douglas function represents a relationship in which the proportions of labor and capital with respect to the total product are constant.
\end{abstract}