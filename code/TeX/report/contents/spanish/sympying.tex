\section*{PythonTeX: py}
% eingebetteter Python-Aufruf
Wissen Sie, dass $2^{65} = \py{2**65}$?

\section*{PythonTeX: pycode/pyblock-Umgebung, printpythontex, ...}
\begin{pyblock}
# Aufbau einer tabular-Umgebung in einer Schleife
# Python-Code wird ausgegeben
anfang, ende = 1, 30
print(r"\begin{tabular}{r|r}")
print(r"$m$ & $2^m$ \\ \hline")
for m in range(anfang, ende + 1):
	print(m, "&", 2**m, r"\\")
print(r"\end{tabular}")
\end{pyblock}
\printpythontex % Ausgabe des Blocks

\newpage

% aus Mertz, Slough 2013 - A Gentle Introduction to PythonTeX
\section*{PythonTeX: pythontexcustomcode, sympy, def, Schleife, Primzahl}
\begin{pythontexcustomcode}{py}
from sympy import prime		# symb. Mathematik, hier Primzahlen

def Primzahlen(n):				# Definition einer Python-Funktion
	for i in range(1, n):		# Annahme n >= 3
		print(prime(i), " ")	# nächste Primzahl
	print("und ", prime(n))	# letzte Primzahl
\end{pythontexcustomcode}

Die ersten 1000 Primzahlen sind \pyc{Primzahlen(1000)}.
\newpage

% aus Mertz, Slough 2013 - A Gentle Introduction to PythonTeX

\section*{PythonTeX: pyblock, printpythontex, sympy, Binome, ...}

\begin{sympyblock}
from sympy import *	# symbolische Mathematik
var("a, b")			# sympy-Variablen
Binome = []			# Liste für Binomi-Ausdrücke vorbesetzt

for m in range(1, 10):
	Binome.append((a + b)**m)	# Binomi-Ausdrücke erzeugen

print(r"\begin{align*}")	# Tabelle mit align*-Umgebung
for expr in Binome:			# SChleife über alle Binome
	print(latex(expr), "&=", latex(expand(expr)), r"\\")
print(r"\end{align*}")
\end{sympyblock}

\printpythontex

\section*{PythonTeX: pyblock, sympy, Gleichungssystem}

\begin{pyblock}
import sympy as sy	# symbolische Mathematik
h, z, e = sy.symbols('h z e')	# sympy-Variablen initiieren
gls = [			# Gleichungssystem formulieren
sy.Eq(z + h + e, 18),
sy.Eq(h - 6, 2 * z),
sy.Eq(e - 6, 3 * z),
]

ergebnis = sy.solve(gls)	# Gleichungssystem lösen
for f in ergebnis:	# Lösung ausgeben
	print(f, ":", ergebnis[f], r"\\")
\end{pyblock}
\printpythontex	% letzten pyblock ausgeben

% Poore 2013 - PythonTeX: Reproducible Documents with PythonTeX
\section*{PythonTeX: sympy, sympyblock, printpythontex, Ableitung, ...}

\begin{sympyblock}
from sympy import *
x = symbols('x')	# sympy-Variable

print(r'\begin{align*}')
for funk in [sin(x), sinh(x), csc(x)]:	# zu untersuchende Funktionen
	links = Derivative(funk, x)	# Ableitung, formal
	rechts = Derivative(funk, x).doit()	# Ableitung ausführen
	gl = latex(links) + '&=' + latex(rechts) + r'\\'
	print(gl.replace('d', r'\mathrm{d} ')) # d austauschen
print(r'\end{align*}')
\end{sympyblock}
\printpythontex