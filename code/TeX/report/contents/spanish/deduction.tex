Las constantes $\alpha$ y $\beta$ tiene un significado económico de acuerdo a su valor.

\begin{itemize}
	\item $\alpha+\beta=1$: la función de producción tiene vueltas a escala constante (cambios en la salida subsecuente a un cambio proporcional en las entradas)
	\item $\alpha+\beta<1$: la función de producción tiene vueltas a escala que disminuyen.
	\item $\alpha+\beta>1$: la función de producción tiene vueltas a escala que aumentan.
\end{itemize}

\subsection{Deducción algebraica de la función de producción de Cobb-Douglas}

Dentro de los supuestos básicos de la función de producción Cobb-Douglas, se tiene:
\begin{itemize}
	\item Si la mano de obra o capital se reduce, la prodicción también se reducen en la misma propducción.
	\item La productividad marginal de la mano de obra es proporcional a la cantidad de producción por unidad de mano de obra.
	\item La productividad marginal del capital es proporcional a la cantidad de producción por unidad de capital.
\end{itemize}
Con base a dichas suposiciones, se plantean las ecuaciones diferenciales relacionadas con este comportamiento:
\begin{align}
\frac{\partial P}{\partial L}
&=\alpha\frac{P}{L}\label{eq:margL}\\
\frac{\partial P}{\partial K}
&=\beta\frac{P}{K}\label{eq:margK}
\end{align}

En relación a las ecuaciones~\eqref{eq:margL} y ~\eqref{eq:margK} se puede decir que:

\begin{align}
K\frac{\partial P}{\partial K}
&=\beta P\label{eq:margLL}\\
L\frac{\partial P}{\partial L}
&=\alpha P\label{eq:margKK}
\end{align}
Sumando las ecuaciones~\eqref{eq:margLL} y ~\eqref{eq:margKK}, sería
\begin{align}
L\frac{\partial P}{\partial L}+K\frac{\partial P}{\partial K}
&=\alpha P+\beta P\label{eq:margLLL}\\
L\frac{\partial P}{\partial L}+K\frac{\partial P}{\partial K}
&=\left(\alpha+\beta\right)P\label{eq:margKKK}
\end{align}
Haciendo $r=a+b$, entonces
\begin{equation}
L\frac{\partial P}{\partial L}+K\frac{\partial P}{\partial K}=rP
\end{equation}
La ecuación~\eqref{eq:margLL} es equivalente al teorema de Euler para funciones homogéneas, lo que indica que si $r=1$, entonces se tendrá una ecuación homogénea de grado $1$ y
\begin{equation}
L\frac{\partial P}{\partial L}+K\frac{\partial P}{\partial K}=P\left(L,K\right)
\end{equation}
La ecuación~\eqref{eq:margL} proporciona la productividad marginal de la mano de obra. Como esta ecuación es una ecuación diferencial ordinaria, la solución la hallamos separando variables e integrando. Así, obtenemos
\begin{equation}
\ln\left(P\right)+c_{1}=\alpha\ln\left(L\right)+g\left(K\right)+c_{2}.
\end{equation}
O equivalentemente,
\begin{equation}
\ln\left(P\right)=\alpha\ln\left(L\right)+g\left(K\right)+C
\end{equation}
\begin{equation}\label{eq:exp}
P=e^{\ln\left(L\right)^{\alpha}}e^{g\left(K\right)}e^{C}
\end{equation}
Haciendo $A=e^{C}$ y $h\left(K\right)=e^{g\left(K\right)}$ la ecuación~\eqref{eq:exp} se transforma:
\begin{equation}
P=AL^{\alpha}h\left(k\right).
\end{equation}
Se sabe que:
\begin{equation}
\frac{\partial P}{\partial K}=\beta\frac{P}{K}
\end{equation}
Derivando parcialmente la función encontrada en el procedimiento anterior y reemplazando:
\begin{equation}
\frac{\partial P}{\partial K}=AL^{\alpha}h\left(K\right)
\end{equation}
\begin{equation}
\beta\frac{P}{K}=AL^{\alpha}h\left(K\right)
\end{equation}
\begin{equation}
\beta\frac{AL^{\alpha}h\left(K\right)}{K}=AL^{\alpha}h\left(K\right)
\end{equation}
La cual se convierte en una ecuación diferencial ordinaria:
\begin{equation}\label{eq:ode}
h^{\prime}\left(K\right)-\beta\frac{h\left(K\right)}{K}=0.
\end{equation}
La solución de esta ecuación diferencial es $h\left(K\right)=K^{\beta}$. Lo cual se verifica fácilmente, ya que al reemplazar en la ecuación anterior se obtiene una identidad. Luego,
\begin{align}
h\left(K\right)
&=K^{\beta}\\
h^{\prime}\left(K\right)
&=\beta K^{\beta-1}
\end{align}
Reemplazando en la ecuación~\eqref{eq:ode}
\begin{align*}
\beta K^{\beta-1}-\frac{\beta K^{\beta}}{K}
&=0\\
\frac{\beta K^{\beta}}{K}
&=\frac{\beta K^{\beta}}{K}
\end{align*}
Realizando la sustitución $y=h\left(K\right)$ se tiene $\frac{dy}{dK}=h^{\prime}\left(K\right)$.

Reemplazando en la ecuación~\eqref{eq:ode}
\begin{equation}
\frac{dy}{dK}-\beta\frac{y}{K}=0
\end{equation}
Separando variables e integrando obtenemos,
\begin{equation}
\ln\left(y\right)+c_{1}=\beta\ln\left(K\right)+c_{2}
\end{equation}
\begin{align*}
\ln\left(y\right)
&=\beta\ln\left(K\right)+c\iff e^{\ln\left(y\right)}=e^{\left(\beta\ln\left(K\right)+c\right)}\iff y=e^{\left(\ln\left(K\right)^{\beta}+c\right)}\\
y&=K^{\beta}e^{c}\iff y=cK^{\beta}\\
h\left(K\right)&=cK^{\beta}
\end{align*}
Luego de encontrar $h\left(K\right)$, tenemos que \[ b=Ac,\quad P=bL^{\alpha}h\left(K\right)\iff P=bL^{\alpha}K^{\beta} \] Y dentro de la suposición de la función de producción se sabe que $\alpha+\beta=1$, por lo tanto
\begin{equation}\label{eq:P}
P=bL^{\alpha}K^{1-\alpha}
\end{equation}
\subsection{Regresión lineal de la función de Cobb-Douglas para un caso en particular}

Para poder hallar los valores de $\alpha$, $\beta$ y $b$ se debe linealizar la función de producción.

La ecuación~\eqref{eq:P} la podemos escribir como $\frac{P}{K}=bL^{\alpha}K^{-\alpha}$ de donde
\begin{align*}
\ln\left(\frac{P}{K}\right)
&=\ln\left(b\left(\frac{L}{K}\right)^{\alpha}\right)\iff\ln\left(\frac{P}{K}\right)=\ln\left(b\right)+\ln\left(\frac{L}{K}\right)^{\alpha}\\
\ln\left(\frac{P}{K}\right)
&=\ln\left(\frac{P}{K}\right))=\ln\left(b\right)+\alpha\ln\left(\frac{L}{K}\right)
\end{align*}